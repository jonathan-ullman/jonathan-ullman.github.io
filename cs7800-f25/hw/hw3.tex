\documentclass[11pt]{article}


\newcommand{\E}{\mathbb{E}}
\usepackage[table,xcdraw]{xcolor}
% set this flag to 0 to remove comments
\def\comments{1}

\usepackage{amsmath,amssymb,amsthm}
\usepackage{bbm}
\usepackage{paralist}
\usepackage[linesnumbered,ruled,vlined]{algorithm2e}
\usepackage{authblk}
	\renewcommand{\Authsep}{\qquad}
	\renewcommand{\Authand}{\qquad}
	\renewcommand{\Authands}{\qquad}
	\renewcommand\Affilfont{\itshape\small}
\usepackage[left=1.25in,right=1.25in,top=1.25in,bottom=1.25in]{geometry}
\usepackage[bookmarks=false]{hyperref}
    \hypersetup{
        linktocpage=true,
        colorlinks=true,				
        linkcolor=DarkBlue,				
        citecolor=DarkBlue,				
        urlcolor=DarkBlue,			
    }
\usepackage[tt=false]{libertine}
    \usepackage[libertine]{newtxmath}
    \usepackage[T1]{fontenc}
    \renewcommand{\baselinestretch}{1.00}
\usepackage{lipsum}
\usepackage{microtype}
\usepackage{multirow}
\usepackage{enumitem}
\usepackage{nicefrac}
\usepackage{tikz}
	\usetikzlibrary{positioning}
	\definecolor{DarkGreen}{rgb}{0.2,0.6,0.2}
	\definecolor{DarkRed}{rgb}{0.6,0.2,0.2}
	\definecolor{DarkBlue}{rgb}{0.2,0.2,0.6}
	\definecolor{DarkPurple}{rgb}{0.4,0.2,0.4}   
\usepackage{url}
\usepackage{verbatim}
\usepackage{wrapfig}
\usepackage{framed}
% \usepackage{ulem}
% \normalem

   
% comments
\setlength\marginparwidth{62pt}
\setlength\marginparsep{5pt}
\ifnum\comments=1
    \newcommand{\mynote}[2]{{\marginpar{\color{#1}\sf \tiny #2}}}
    \newcommand{\mynoteinline}[2]{{\color{#1} \sf \small #2}}
\else
\newcommand{\mynote}[2]{}
    \newcommand{\mynoteinline}[2]{}
\fi
\newcommand{\jnote}[1]{\mynote{blue}{JU: #1}}
\newcommand{\as}[1]{\mynote{red}{ADS: #1}}
\newcommand{\anote}{\as}
\newcommand{\asinline}[1]{\mynoteinline{red}{ADS: #1}}

\renewcommand{\epsilon}{\eps}

% fixing left-right spacing
\let\originalleft\left
\let\originalright\right
\renewcommand{\left}{\mathopen{}\mathclose\bgroup\originalleft}
\renewcommand{\right}{\aftergroup\egroup\originalright}

% math macros
\newcommand{\ex}[2]{{\ifx&#1& \mathbb{E} \else \underset{#1}{\mathbb{E}} \fi \left(#2\right)}}
\newcommand{\pr}[2]{{\operatorname*{\mathbb{P}}_{#1} \paren{#2}}}
\newcommand{\var}[2]{{\ifx&#1& \mathrm{Var} \else \underset{#1}{\mathrm{Var}} \fi \left(#2\right)}}

\newcommand{\mypar}[1]{\medskip\textbf{#1}}

\newcommand{\from}{:}
\newcommand{\poly}{\mathrm{poly}}
\newcommand{\polylog}{\mathrm{polylog}}
\newcommand{\eps}{\varepsilon}

\newcommand{\ind}{\mathbb{I}}
\newcommand{\pmo}{\{\pm 1\}}
\newcommand{\zo}{\{0,1\}}
\newcommand{\bit}[1]{{\zo^{#1}}}
\newcommand{\cond}[1]{\vert_{#1}}
\newcommand{\norm}[2]{\|#1\|_{#2}}
\newcommand{\abs}[1]{{\left | {#1} \right|}}
\newcommand{\half}{\frac{1}{2}}
\newcommand{\nicehalf}{\nicefrac{1}{2}}
\newcommand{\set}[1]{\left\{ #1 \right\}}
\newcommand{\cardset}[1]{\left| \set{#1} \right|}
\newcommand{\paren}[1]{{\left ( {#1} \right)}}
\newcommand{\bparen}[1]{{\big( {#1} \big)}}
\newcommand{\Bparen}[1]{{\Big( {#1} \Big)}}
\newcommand{\bracket}[1]{{\left [ {#1} \right]}}
\newcommand{\ip}[1]{{\left \langle {#1} \right\rangle }}

\newcommand{\distance}{\mathrm{d}}
\newcommand{\dtv}{\distance_{\mathrm{TV}}}
\newcommand{\dkl}{\distance_{\mathrm{KL}}}
\newcommand{\dhe}{\distance_{\mathrm{H^2}}}
\newcommand{\dcs}{\distance_{\mathrm{\chi^2}}}

\DeclareMathOperator*{\argmin}{arg\,min}
\DeclareMathOperator*{\argmax}{arg\,max}

\newcommand{\N}{\mathbb{N}}
\newcommand{\R}{\mathbb{R}}
\newcommand{\Z}{\mathbb{Z}}

\newcommand{\cA}{\mathcal{A}}
\newcommand{\cB}{\mathcal{B}}
\newcommand{\cC}{\mathcal{C}}
\newcommand{\cD}{\mathcal{D}}
\newcommand{\cE}{\mathcal{E}}
\newcommand{\cF}{\mathcal{F}}
\newcommand{\cG}{\mathcal{G}}
\newcommand{\cH}{\mathcal{H}}
\newcommand{\cI}{\mathcal{I}}
\newcommand{\cJ}{\mathcal{J}}
\newcommand{\cK}{\mathcal{K}}
\newcommand{\cL}{\mathcal{L}}
\newcommand{\cM}{\mathcal{M}}
\newcommand{\cN}{\mathcal{N}}
\newcommand{\cO}{\mathcal{O}}
\newcommand{\cP}{\mathcal{P}}
\newcommand{\cQ}{\mathcal{Q}}
\newcommand{\cR}{\mathcal{R}}
\newcommand{\cS}{\mathcal{S}}
\newcommand{\cT}{\mathcal{T}}
\newcommand{\cU}{\mathcal{U}}
\newcommand{\cV}{\mathcal{V}}
\newcommand{\cW}{\mathcal{W}}
\newcommand{\cX}{\mathcal{X}}
\newcommand{\cY}{\mathcal{Y}}
\newcommand{\cZ}{\mathcal{Z}}


\newcommand{\bc}{\mathbf{c}}
\newcommand{\bp}{\mathbf{p}}
\newcommand{\bw}{\mathbf{w}}
\newcommand{\by}{\mathbf{y}}

\newcommand{\defeq}{\stackrel{{\mbox{\tiny def}}}{=}}

\newtheorem{thm}{Theorem}
    \newtheorem{clm}[thm]{Claim}
    \newtheorem{lem}[thm]{Lemma}
    \newtheorem{prop}[thm]{Proposition}
    \newtheorem{cor}[thm]{Corollary}
    \newtheorem{fact}[thm]{Fact}
    \newtheorem{claim}[thm]{Claim}
\theoremstyle{definition}
    \newtheorem{defn}[thm]{Definition}
    \newtheorem{example}[thm]{Example}
    \newtheorem{rem}[thm]{Remark}
    \newtheorem{exer}[thm]{Exercise}

\newcommand{\HWtitle}[2]{\begin{figure}[t!]{\bfseries \Large \color{DarkBlue}  \noindent CS7800: Advanced Algorithms \hfill Fall 2025} \\[0.2em] {\bfseries \Large \color{DarkBlue} Homework #1: Due {#2}} \\[1em] {\bfseries \large Jonathan Ullman}\\[1ex] \end{figure}}

\newcommand{\HWsoltitle}[2]{\begin{figure}[t!]{\bfseries \Large \color{DarkBlue}  \noindent CS7800: Advanced Algorithms \hfill Fall 2025} \\[0.2em] {\bfseries \Large \color{DarkBlue} Homework #1 Solutions} \\[1em] {\bfseries \large Jonathan Ullman}\\[1ex] \end{figure}}

\begin{document}

%%%%%%%%%%%%%%%%%%%%%%%%%%%%%%%%%%%%
%%%%%%%%%%%%  Title  %%%%%%%%%%%%%%%
%%%%%%%%%%%%%%%%%%%%%%%%%%%%%%%%%%%%

\HWtitle{3}{Friday October 2, 2025}

%%%%%%%%%%%%%%%%%%%%%%%%%%%%%%%%%%%%
%%%%%%%%%%  Beginning  %%%%%%%%%%%%%
%%%%%%%%%%%%%%%%%%%%%%%%%%%%%%%%%%%%

\renewcommand{\labelenumii}{{\bfseries \em \arabic{enumi}.\arabic{enumii}}}
\newcommand{\problemitem}{\renewcommand{\labelenumi}{{\bfseries \em Problem \arabic{enumi}}}\item}
\newcommand{\solutionitem}{\renewcommand{\labelenumi}{{\bfseries \em Solution \arabic{enumi}}}\addtocounter{enumi}{-1}\item}

\noindent\textbf{\color{blue} \Large Assigned Problems (Collected and Graded)}
\begin{enumerate}[leftmargin=0pt]
%%%%%%%%%%%%%%%%%%%%%%%%%%%%%%%%%%%%%%%%%%%%%%%%%%%%%%
%%%%%%%%%%%%%%%%%%%%%%%%%%%%%%%%%%%%%%%%%%%%%%%%%%%%%%
%%%%%%%%%%%%%%%%%%%%%%%%%%%%%%%%%%%%%%%%%%%%%%%%%%%%%%
%%%%%%%%%%%%%%%%%%%%%%%%%%%%%%%%%%%%%%%%%%%%%%%%%%%%%%
\problemitem Let $G = (V,E,\{c_e\})$ be a network with integer edge capacities.  We say that an edge $e$ is \emph{flow-enhancing} if increasing
its capacity by 1 also increases the value of the maximum flow in $G$. Similarly, an edge $s$ is
\emph{flow-reducing} if decreasing its capacity by 1 also decreases the value of the maximum flow
in $G$.
\begin{enumerate}[leftmargin=0pt, itemsep=3ex]
\item Does every network $G$ have at least one flow-enhancing edge?  Either prove that the answer is yes or give  a counterexample.

\item Describe and analyze an algorithm to find all flow-enhancing edges in $G$, given both $G$ and a
maximum flow in $G$ as input.\footnote{Hint: Suppose $f$ is a maximum flow in $G$, and $G_{f}$ is the residual graph corresponding to $f$.  What must be true about $G_f$ if the edge $e$ is flow-enhancing/flow-reducing? \label{footnote1}}  Your algorithm should run in at most $O(m^2)$ time\footnote{Note that there is a trivial algorithm that recomputes the maximum flow after increasing/decreasing each of the $m$ edges, which takes $m \times O(nm) = O(nm^2)$ time.  Your algorithm should run faster than this baseline. \label{footnote2}} for full credit, but for a challenge try to do better.

\item Prove that an edge is flow-enhancing \emph{if and only if} it appears in \emph{every} minimum cut of the graph $G$.

\item Does every network $G$ have at least one flow-reducing edge?  Either prove that the answer is yes or give  a counterexample.

\item Describe and analyze an algorithm to find all flow-reducing edges in $G$, given both $G$ and a
maximum flow in $G$ as input.\textsuperscript{\ref{footnote1}}  Your algorithm should run in at most $O(m^2)$ time\textsuperscript{\ref{footnote2}} for full credit, but for a challenge try to do better.
\end{enumerate}
%%%%%%%%%%%%%%%%%%%%%%%%%%%%%%%%%%%%%%%%%%%%%%%%%%%%%%
%%%%%%%%%%%%%%%%%%%%%%%%%%%%%%%%%%%%%%%%%%%%%%%%%%%%%%

\vspace{3ex}
\noindent\textbf{\color{red} \Large Optional Problems (Collected and Graded)}
%%%%%%%%%%%%%%%%%%%%%%%%%%%%%%%%%%%%%%%%%%%%%%%%%%%%%%
%%%%%%%%%%%%%%%%%%%%%%%%%%%%%%%%%%%%%%%%%%%%%%%%%%%%%%
%%%%%%%%%%%%%%%%%%%%%%%%%%%%%%%%%%%%%%%%%%%%%%%%%%%%%%
%%%%%%%%%%%%%%%%%%%%%%%%%%%%%%%%%%%%%%%%%%%%%%%%%%%%%%
\problemitem {\color{red} I am not assigning this problem because it's a bit open ended and there are some fiddly technical details in order to get a perfectly correct solution, but I think it's a really good problem to work on.}

Suppose that instead of capacities, we consider networks where each edge $e$ has some non-negative, integer-valued \emph{demand} $d(e) \geq 0$.  We say that an $(s,t)$-flow $f$ is \emph{feasible} if $f(e) \geq d(e)$ for every edge $e$, in addition to satisfying the flow-conservation constraints.  A natural problem in this setting is to find a feasible flow of \emph{minimum} value.
\begin{enumerate}[leftmargin=0pt, itemsep=1ex]
	\item Describe an efficient algorithm to compute a \emph{feasible} $(s,t)$-flow.  That is, one that satisfies flow conservation and demand constraints, but is not necessarily of minimum value.  Justify your algorithm's correctness, and analyze its running time.
	
	\item Suppose you have access to a subroutine $\textsc{MaxFlow}$ that solves the maximum $s$-$t$-flow problem. Describe an efficient algorithm to compute a minimum flow in a network with edge demands.  Your algorithm should call $\textsc{MaxFlow}$ exactly once.\footnote{\textbf{Hint:} Start with the feasible $(s,t)$-flow your algorithm finds in 4.1 and try to \emph{remove} as much flow as possible while still satisfying the demands.}  Justify that your algorithm is correct and analyze its running time, including the time required to execute $\textsc{MaxFlow}$.

    \item In this problem you might guess that there is a \emph{min-flow/max-cut} theorem, where the value of the minimum flow is equal to the demand of the maximum cut.  Show that this theorem is \emph{false}.  That is, draw a graph where there is a cut with some total demand $D$ but also a flow satisfying all the demand constraints with value $F < D$.
    
	\item State and prove an analogue of the max-flow/min-cut theorem for this setting.  That is, express the value of the minimum flow in terms of something about the cuts in the graph.
\end{enumerate}

\end{enumerate}
\end{document}
